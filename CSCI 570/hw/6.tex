\documentclass{article}

\usepackage[noend]{algpseudocode}
\usepackage{mathtools}

\title{Answers to CSCI 570 - Fall 2021 - HW 6}
\author{Shangning Xu}

\begin{document}

\maketitle

\section*{Graded Problems}

\subsection*{Problem 1}

\begin{algorithmic}[1]
    \State // $r[i]$ is the maximum revenue of selling a length-$i$ rod.
    \State Allocate the array $r[0..N]$.
    \For{$i \gets 1$ \textbf{to} $N$}
        \State $r[i] \gets p_i$
        \For{$j \gets 1$ \textbf{to} $i - 1$}
            \State $r[i] \gets \max(r[i], p_j + r[i - j])$
        \EndFor
    \EndFor
    \State \Return $r[N]$
\end{algorithmic}

The outer loop runs for $N$ iterations. In the $i$th iteration, the inner loop will run for $i$ iterations, giving a total of $N(1 + N)/2$ iterations. Therefore, the time complexity is $\Theta(N^2)$.

\subsection*{Problem 2}

\begin{algorithmic}[1]
    \State $l \gets 1$
    \State $r \gets N$
    \State $d \gets 0$ \Comment{Difference in score.}
    \State Sum all marbles' values and store it in $s$.
    \State $t \gets \mathrm{TRUE}$ \Comment{TRUE if it's Tommy's turn.}
    \While{$l \le r$}
        \If{$m_l < m_r$} \Comment{Choose the marble with lower value.}
            \State $s \gets s - m_l$
            \State $l \gets l + 1$
        \Else
            \State $s \gets s - m_r$
            \State $r \gets r - 1$
        \EndIf
        \If{$t$}
            \State $d \gets d + s$
            \State $t \gets \mathrm{FALSE}$
        \Else
            \State $d \gets d - s$
            \State $t \gets \mathrm{TRUE}$
        \EndIf
    \EndWhile
    \State \Return $d$
\end{algorithmic}

As the loop only runs for $N$ iterations and the time for summing marble values is $\Theta(N)$, the time complexity is $\Theta(N)$.

\subsection*{Problem 3}

\begin{algorithmic}[1]
    \State // Compute $l[i]$, the length of the longest line of increasing heights that
    \State // ends at the $i$th member and is formed by pulling out members.
    \State $l[1] \gets 1$
    \For{$i \gets 2$ \textbf{to} $n$}
        \State $l[i] \gets 0$
        \For{$j \gets 1$ \textbf{to} $i - 1$}
            \If{$r_j < r_i$}
                \State $l[i] \gets \max(l[i], l[j])$
            \EndIf
        \EndFor
        \State $l[i] \gets l[i] + 1$
    \EndFor
    \State // Compute $r[i]$, the length of the longest line of decreasing heights that
    \State // starts at the $i$th member and is formed by pulling out members.
    \State $r[n] \gets 1$
    \For{$i \gets n - 1$ \textbf{downto} $1$}
        \State $r[i] \gets 0$
        \For{$j \gets i + 1$ \textbf{to} $n$}
            \If{$r_i < r_j$}
                \State $r[i] \gets \max(r[i], r[j])$
            \EndIf
        \EndFor
        \State $r[i] \gets r[i] + 1$
    \EndFor
    \State $s \gets 0$ \Comment{Length of the longest formation}
    \For{$i \gets 1$ \textbf{to} $n$}
        \State $s \gets \max(s, l[i] + r[i])$
    \EndFor
    \State $s \gets s - 1$
    \State \Return $n - s$
\end{algorithmic}

The loops to compute the arrays $l$ and $r$ each run for $\Theta(n^2)$ iterations, and the last loop to compute $n$ runs for $n$ iterations, giving a time complexity of $\Theta(n^2)$.

\subsection*{Problem 4}

Assume $f$ to be the total fee for a pair of buy and sell actions.

\begin{algorithmic}[1]
    \State // $b$ is the maximum profit one can make when they start trading on
    \State // the ($n, n - 1, \dots, 1$)th day and don't own any stock.
    \State // $s$ is the maximum profit one can make when they start trading on
    \State // the ($n, n - 1, \dots, 1$)th day and own one unit of stock.
    \State $b_\textrm{prev} \gets s_\textrm{prev} \gets 0$
    \For{$i \gets n$ \textbf{down to} 1}
        \State $b \gets \max(b_\textrm{prev}, s_\textrm{prev} - prices[i])$
        \State $s \gets \max(s_\textrm{prev}, b_\textrm{prev} + prices[i] - f)$
        \State $b_\textrm{prev} \gets b$
        \State $s_\textrm{prev} \gets s$
    \EndFor
    \State \Return $b$
\end{algorithmic}

The inner loop will run for $N$ iterations, giving a time complexity of $\Theta(N)$.

\section*{Practice Problems}

\subsection*{Problem 5}

Let $p_i$ be the optimal value that can be obtained with a knapsack of capacity $i$. We have
\[
    p_i = \begin{dcases*}
        0 & if $i < \min_{j = 1, \dots, n} w_i$,\\
        \max_{j = 1, \dots, n} (p_{i - w_j} + v_j) & otherwise.\\
    \end{dcases*}
\]

To compute $p_W$, we create and fill the array $p[0..W]$ according to the recurrence above, from index $\min_{i = 1, \dots, n} w_i$ to $W$.

\subsection*{Problem 6}

Let $s_{i, j}$ denote the maximum number of coins obtained by bursting balloon $i$ to $j - 1$ and $v_i = nums[i]$. We have
\[
    s_{i, j} = \begin{dcases*}
        0 & if $i = j$,\\
        \max_{k = i, \dots, j - 1} (s_{i, k} + s_{k + 1, j} + v_{i - 1}v_kv_j) & otherwise.
    \end{dcases*}
\]

\subsection*{Problem 7}

Denote the substring $y_iy_{i + 1}{\cdots}y_{j - 1}$ as $y_{i, j}$ and let $p_j$ be the maximum quality for the segmentation of string $y_{1, j}$. We have
\[
    p_j = \begin{dcases*}
        0 & if $j = 1$,\\
        \max_{i = 1, \dots, j - 1}(p_j + quality(y_{i, j})) & otherwise.
    \end{dcases*}
\]

\subsection*{Problem 8}

Let $p_i$ be the minimum sum of squares of slack when formatting the set of words $W_i = \{w_i, w_{i + 1}, \dots, w_n\}$ and $l(W_{i, j})$ be the length of the line formed by words in $W_{i, j} = \{w_i, w_{i + 1}, \dots, w_j\}$, that is
\[
    l(W_{i, j}) = \sum_{k = i}^{j - 1} (c_k + 1) + c_j.
\]

We have
\[
    p_i = \begin{dcases*}
        0 & if $i = n + 1$,\\
        \min_{j: l(W_{i, j}) \le L} \left[(L - l(W_{i, j}))^2 + p_{j + 1}\right] & otherwise.
    \end{dcases*}
\]

\end{document}
