\documentclass{article}

\usepackage{mathtools}

\title{Answers to CSCI 570 - Fall 2021 - HW 12}
\author{Shangning Xu}

\begin{document}

\maketitle

\section*{Graded Problems}

\subsection*{Problem 1}

\begin{align*}
    \textrm{maximize}&\quad \sum_{i = 1}^9 \alpha_i(C_i - S_i),\\
    \textrm{subject to}&\quad 0 \le S_i \le \frac{C_i}{2}, i = 1, \dots, 9.
\end{align*}

\subsection*{Problem 2}

We present the following algorithm for the triangle removal problem: for each combination of three vertices from $V$, remove all three edges between the three vertices if they form a triangle. The time complexity is $O(n(n - 1)(n - 2)/6)$, so the algorithm is strongly polynomial.

Proving that the algorithm has an approximation ratio of 3 is similar to proving that the approximation algorithm for vertex cover has an approximation ratio of 2. Let $A$ be a set of triangles selected by our algorithm, where a triangle is represented by a triple of three edges sharing endpoints, then no triangle in $A$ share edges. Let $S^*$ be an optimal subset of edges to remove and $S$ be a set of edges returned by our algorithm. Because removing all edges from $S^*$ leads to complete removal of all triangles in $G$, for each triangle in $A$, at least one edge from the triangle must be in $S^*$, yielding
\[
    |A| \le |S^*|.
\]

Since three edges are added to $S$ for each triangle in $A$, we have $|S| = 3|A|$. Combining the equations gives
\[
    |S| = 3|A| \le 3|S^*|.
\]

\subsection*{Problem 3}

We assume that the objective is to maximize the total profit. Let $x_{ij}$ be the number of units of product $j$ produced by plant $i$. In the following formulation, ``$*$'' denotes element-wise matrix multiplication, inequality operators in constraints denote element-wise inequality, and \textbf{1} rendered in bold is a column vector of all ones of appropriate length.

\begin{align*}
    \textrm{maximize}\quad &\left(\begin{bmatrix}
        10 & 8 & 6 & 9\\
        18 & 20 & 15 & 17\\
        15 & 16 & 13 & 17
    \end{bmatrix}
    * \begin{bmatrix}
        x_{11} & x_{21} & x_{31} & x_{41}\\
        x_{12} & x_{22} & x_{32} & x_{42}\\
        x_{13} & x_{23} & x_{33} & x_{43}
    \end{bmatrix}\right)
    \mathbf{1},\\
    \textrm{subject to}\quad &\left(\begin{bmatrix}
        5 & 6 & 13\\
        7 & 12 & 14\\
        4 & 8 & 9\\
        10 & 15 & 17
    \end{bmatrix}
    * \begin{bmatrix}
        x_{11} & x_{12} & x_{13}\\
        x_{21} & x_{22} & x_{23}\\
        x_{31} & x_{32} & x_{33}\\
        x_{41} & x_{42} & x_{43}
    \end{bmatrix}\right) \mathbf{1}
    \le 35 \cdot 60 \cdot \mathbf{1},\\
    & x_{ij} \ge 0, i = 1, \dots, 3, j = 1, \dots, 4.
\end{align*}

\section*{Ungraded Problems}

\subsection*{Problem 1}

Let $\alpha_i$ be the fraction of object $i$ that will be put in the knapsack. We have
\begin{align*}
    \textrm{maximize}\quad &\sum_{i = 1}^n \alpha_im_i,\\
    \textrm{subject to}\quad &\sum_{i = 1}^n \alpha_im_i \le W,\\
    &\sum_{i = 1}^n \alpha_iv_i = V,\\
    &0 \le \alpha_i \le 1, i = 1, \dots, n.
\end{align*}

\end{document}
