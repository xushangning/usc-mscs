\documentclass{article}

\usepackage{mathtools}

\title{Answers to CSCI 570 - Fall 2021 - HW 12}
\author{Shangning Xu}

\begin{document}

\maketitle

\section*{Graded Problems}

\subsection*{Problem 1}

\begin{align*}
    \textrm{maximize}&\quad \sum_{i = 1}^9 \alpha_i(C_i - S_i),\\
    \textrm{subject to}&\quad 0 \le S_i \le \frac{C_i}{2}, i = 1, \dots, 9,\\
    &\sum_{i = 1}^9 S_i = 800.
\end{align*}

\subsection*{Problem 2}

We present the following algorithm for the triangle removal problem: for each combination of three vertices from $V$, remove all three edges between the three vertices if they form a triangle. The time complexity is $O(n(n - 1)(n - 2)/6)$, so the algorithm is strongly polynomial.

Proving that the algorithm has an approximation ratio of 3 is similar to proving that the approximation algorithm for vertex cover has an approximation ratio of 2. Let $A$ be a set of triangles selected by our algorithm, where a triangle is represented by a triple of three edges sharing endpoints, then no triangle in $A$ share edges. Let $S^*$ be an optimal subset of edges to remove and $S$ be a set of edges returned by our algorithm. Because removing all edges from $S^*$ leads to complete removal of all triangles in $G$, for each triangle in $A$, at least one edge from the triangle must be in $S^*$, yielding
\[
    |A| \le |S^*|.
\]

Since three edges are added to $S$ for each triangle in $A$, we have $|S| = 3|A|$. Combining the equations gives
\[
    |S| = 3|A| \le 3|S^*|.
\]

\subsection*{Problem 3}

We assume that the objective is to maximize the total profit. Define $X = (x_{ij})$, where $x_{ij}$ is the number of units of product $i$ produced by plant $j$. Let $A$ be the production time per unit matrix and $P$ be the profit per unit matrix, with elements arranged in exactly the same way as the tables given in the question.

In the following formulation, ``$*$'' denotes element-wise matrix multiplication, inequality operators in constraints denote element-wise inequality, and \textbf{1} rendered in bold is a column vector of all ones of appropriate length.

\begin{align*}
    \textrm{maximize}\quad &\mathbf{1}^\textrm{T}(A * X)\mathbf{1},\\
    \textrm{subject to}\quad &(P * X)^\textrm{T} \mathbf{1} \le 35 \cdot 60 \cdot \mathbf{1},\\
    & X\mathbf{1} \ge [100, 150, 100]^\textrm{T},\\
    & x_{ij} \ge 0, i = 1, \dots, 3, j = 1, \dots, 4.
\end{align*}

\section*{Ungraded Problems}

\subsection*{Problem 1}

Let $\alpha_i$ be the fraction of object $i$ that will be put in the knapsack. We have
\begin{align*}
    \textrm{maximize}\quad &\sum_{i = 1}^n \alpha_im_i,\\
    \textrm{subject to}\quad &\sum_{i = 1}^n \alpha_im_i \le W,\\
    &\sum_{i = 1}^n \alpha_iv_i = V,\\
    &0 \le \alpha_i \le 1, i = 1, \dots, n.
\end{align*}

\subsection*{Problem 2}

Start with empty sets $A$ and $B$ and in each iteration, add any pair of vertices $(u, v)$ to $A$ or $B$ according which choice maximizes $|E(A, B)|$. That is, if
\[
    |E(A \cup \{u\},B \cup \{v\})| \ge |E(A \cup \{v\},B \cup \{u\})|,
\]
then add $u$ to $A$ and $v$ to $B$, vice versa.

Let $N_{Av}$ and $N_{Bv}$ be the number of vertices adjacent to $v$ in $A$ and $B$ before $v$ is added to either $A$ or $B$, respectively. Let $I_{uv}$ be the indicator variable for the edge $(u, v)$: if there is an edge $(u, v)$, $I_{uv} = 1$, and 0 otherwise. For any pair of vertices $(u, v)$ added during an iteration, we have
\[
    \max(N_{Av} + N_{Bu}, N_{Au} + N_{Bv}) + I_{uv} \ge \frac{1}{2}(N_{Av} + N_{Bu} + N_{Au} + N_{Bv} + I_{uv}).
\]

Summing both sides on the pairs of vertices added in each iteration, we have
\begin{align*}
    E_(A, B) &= \sum_{(u, v)} [\max(N_{Av} + N_{Bu}, N_{Au} + N_{Bv}) + I_{uv}]\\
    &\ge \sum_{(u, v)} \frac{1}{2}(N_{Av} + N_{Bu} + N_{Au} + N_{Bv} + I_{uv})\\
    &= \frac{1}{2}|E|,
\end{align*}
which proves that the approximation ratio is 1/2.

\end{document}
