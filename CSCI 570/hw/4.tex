\documentclass{article}

\usepackage[noend]{algpseudocode}
\usepackage{amsthm}

\title{Answers to CSCI 570 - Fall 2021 - HW 4}
\author{Shangning Xu}

\begin{document}

\maketitle

\section*{Section 1: Heaps}

\subsection*{Problem 1}

We use a max-heap $A.max$ and a min-heap $A.min$ to implement the data structure. The max-heap will contain the first half (and the median when the number of total elements is odd) of all elements and min-heap the other half.

\begin{algorithmic}[1]
    \Function{Find-Median}{$A$}
        \If{$n$ is odd}
            \State \Return $A.max[0]$
        \Else
            \State \Return $(A.max[0] + A.min[0])/2$
        \EndIf
    \EndFunction
    \Function{Insert}{$A$, $k$}
        \If{$n = 0$}
            \State MAX-HEAP-INSERT($A.max$, $k$)
        \ElsIf{$n$ is odd}
            \If{$k < A.max[0]$}
                \State $x \gets \textrm{HEAP-EXTRACT-MAX}(A.max)$
                \State MIN-HEAP-INSERT($A.min$, $x$)
                \State MAX-HEAP-INSERT($A.max$, $k$)
            \Else
                \State MIN-HEAP-INSERT($A.min$, $k$)
            \EndIf
        \ElsIf{$k > A.min[0]$}
            \State $x \gets \textrm{HEAP-EXTRACT-MIN}(A.min)$
            \State MAX-HEAP-INSERT($A.max$, $x$)
            \State MIN-HEAP-INSERT($A.min$, $k$)
        \Else
            \State MAX-HEAP-INSERT($A.max$, $k$)
        \EndIf
        \State $n \gets n + 1$
    \EndFunction
\end{algorithmic}

\subsection*{Problem 2}

We will use a min-heap $A$ to track the $k$ largest numbers. The algorithm is as follows:
\begin{algorithmic}[1]
    \While{true}
        \State Fetch a number from the stream and store it in $x$
        \If{$A.heap\textrm{-}size < k$}
            \State MIN-HEAP-INSERT($A$, $x$)
        \ElsIf{$x > A[0]$}
            \State EXTRACT-MIN($A$)
            \State MIN-HEAP-INSERT($A$, $x$)
        \EndIf
    \EndWhile
\end{algorithmic}

\section*{Section 2: MST}

\subsection*{Problem 3}

\begin{proof}
    We will prove by contradiction.

    Assume that $G$ has two minimum spanning trees $T = (V, E)$ and $T' = (V, E')$, where they differ by an edge $(u, v)$. We can further assume, without loss of generality, that $(u, v)$ belongs to $E$ but not $E'$. Because $T'$ is a tree, $u$ and $v$ is connected by some simple path $p$ in $T'$. $p$ can't be present in $T$ as otherwise it would form a cycle with the edge $(u, v)$ and thus must have some missing edge $e$ that belongs to $T'$ but not $T$. Due to cost difference between edges, we can either replace $(u, v)$ with $e$ in $T$ or $e$ with $(u, v)$ in $T'$ and obtain a new spanning tree with less cost, contradicting our assumption that both $T$ and $T'$ are minimum spanning trees.
\end{proof}

\subsection*{Problem 4} 

We present the following algorithm:
\begin{algorithmic}[1]
    \State Use BFS to construct a tree $(V, T)$ in $G$
    \State $S \gets \{e| e \notin T\}$
    \For{$(u, v) \in S$}
        \State Find the path $p$ that connects $u$ and $v$ in $T$
        \State Find the edge $e \in p$ with greatest weight
        \If{$w(u, v) < w(e)$}
            \State $T \gets T \cup \{(u, v)\} - \{e\}$
        \EndIf
        \State $S \gets S - \{(u, v)\}$
    \EndFor
\end{algorithmic}

Because the time complexity of BFS and searching for a cycle is $O(n)$, and that there are at most 9 iterations in the loop, the algorithm runs in $O(n)$.

\section*{Section 3: Shortest Path}

\subsection*{Problem 5}

\begin{enumerate}
    \item False. Consider 3 vertices arranged in a triangle with edge weights 1, 2, 3. There are two shortest paths between the vertices connected by the weight-3 edge.
    \item False. Again, consider 3 vertices arranged in a triangle with edge weights 1, 1, 3 and the shortest path between the vertices connected by the weight-3 edge. The shortest path distance is 2 before the increase and 5 after increasing each edge's weight by $k = 2$. The shortest path distance increases by 3, which is not a multiple of 2.
    \item False. Consider a graph with a cycle that shares an edge $e$ with the simple path from $s$ to $t$. Choose such $k$ that after decreasing $e$'s weight by $k$, the cycle is a negative-weight cycle. As we can traverse the negative-weight cycle for infinite times, the shortest-path weight from $s$ to $t$ is now $-\infty$, a decrease by more than $k$.
    \item False. Again, consider 3 vertices arranged in a triangle with edge weights 2, 2, 3 and the shortest path between the vertices connected by the weight-3 edge. Before the change, the shortest path is the weight-3 edge. After the change, the shortest path is the path composed of two weight-2 edges.
\end{enumerate}

\subsection*{Problem 6}

Let $\delta(s, v)$ for the normal shortest-path distance between $v$ and $s$, and $\delta'(s, v)$ for the shortest-path distance between $v$ and $s$ where one edge on a shortest path is set to zero. We can write the following recurrence:
\[
    \delta'(s, v) = \min_{(u, v) \in E}(\delta'(s, u) + c(u, v), \delta(s, u)).
\]

To compute $\delta'(s, t)$, we first run the Dijkstra algorithm to compute $\delta(s, v)$ for all vertices. Then, we run our modified Dijkstra algorithm with the following changes:
\begin{enumerate}
    \item We maintain $v.d'$ for all vertex $v$, an upper bound for $\delta'(s, v)$, as an array.
    \item We will extract a vertex $u$ with minimum $u.d'$ and add it to the set $S$ of vertices whose $\delta'$ has been computed. Then we perform the following update for all vertex $v$ adjacent to $u$:
    \[
        v.d' \gets \min(v.d', u.d' + c(u, v), \delta(s, u)).
    \]
    \item The algorithm is terminated once the vertex $t$ is added to $S$.
\end{enumerate}

When the algorithm terminates, the value $t.d'$ is the shortest-path weight with one zero-cost edge between $s$ and $t$.

\end{document}
