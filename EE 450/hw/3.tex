\documentclass{article}

\usepackage{enumitem}
\setlist[enumerate, 2]{label=\alph*.}

\usepackage{siunitx}
\usepackage{booktabs}

\title{EE 450 Homework 3}
\author{Shangning Xu}

\begin{document}

\maketitle

\section*{Chapter 6}

\begin{enumerate}
    \item[P18.] First, B must start transmission before the frame transmitted by A reaches B, because the question requires that B begins transmission before A finishes. Otherwise, B will detect signal energy from its adapter and postpone transmission.
    
    Therefore, B starts transmission between $t = 0$ and 325 bit times and the frame reaches A between $t = 325$ and 750 bit times. As a result, A can finish transmitting before it detects that B has transmitted when the size of the frame transmitted by A is in the range of \SIrange{325}{750}{b}.
    
    In the worst case, A transmits a minimum-sized frame of 576 bits and B starts transmission just before the first bit of A's frame reaches B. B's signal reaches A just before $t = 750$ bit times.

    \item[P19.] Consider the example in the question. B schedules its retransmission at $t = 245 + 48 + 512 = 805$ bit times.
    
    A will begin retransmission at $t = 245 + 48 + 245 + 96 = 634$ bit times, because when A detects at $t = 245$ bit times that B has transmitted, A has to wait for bits in the channel (\SI{245}{b} and the 48-bit jam signal) to die down.
    
    Its signal will reach B at $t = 634 + 245 = 879$ bit times. B will refrain from retransmission at its scheduled time A's signal arrives before the scheduled time.
 
    \item[P22.] 
    \begin{tabular}{@{}lllll@{}}
        \toprule
        Question & \begin{tabular}[c]{@{}l@{}}Source\\ MAC\end{tabular} & \begin{tabular}[c]{@{}l@{}}Destination\\ MAC\end{tabular} & \begin{tabular}[c]{@{}l@{}}Source\\ IP\end{tabular} & \begin{tabular}[c]{@{}l@{}}Destination\\ IP\end{tabular} \\ \midrule
        \textit{(i)} & A & \begin{tabular}[c]{@{}l@{}}Right Router's\\ Left NIC\end{tabular} & A & F \\
        \textit{(ii)} & A & \begin{tabular}[c]{@{}l@{}}Right Router's\\ Left NIC\end{tabular} & A & F \\
        \textit{(iii)} & \begin{tabular}[c]{@{}l@{}}Right Router's\\ Right NIC\end{tabular} & F & A & F \\ \bottomrule
    \end{tabular}

    \item[P23.] The maximum total aggregate throughput is \SI{1100}{Mbps}, because every link is fully duplex and hosts connected to the same switch can transfer data in a ring.

    \item[P24.] The maximum total aggregate throughput is now \SI{500}{Mbps}, where two servers can still maintain a bidirectional throughput of \SI{200}{Mbps} while among hosts connected to a departmental hub, there is only one host that can be sending data at a time as hosts connected to a hub form a broadcast LAN.
 
    \item[P25.] The maximum total aggregate throughput is now \SI{100}{Mbps}, because there is only one host that can be sending data at a time as every bit sent will be broadcast to every host in the LAN.
    
    \item[P26.] The switch tables shown below are the state of the table after an event in the given question occurs.
    \begin{enumerate}[label=\textit{(\roman*)}]
        \item The frame is broadcast to all links except the one connected to B, because the switch table doesn't have an entry for E.
        
        \begin{tabular}{@{}lll@{}}
            \toprule
            MAC Address & Interface & Time \\ \midrule
            B & B & \textit{(i)} \\ \bottomrule
        \end{tabular}

        \item The frame is forwarded only to B's link, as the switch table has an entry for B created when the frame sent by B went through the switch.
        
        \begin{tabular}{@{}lll@{}}
            \toprule
            MAC Address & Interface & Time \\ \midrule
            B & B & \textit{(i)} \\
            E & E & \textit{(ii)} \\ \bottomrule
        \end{tabular}

        \item The frame is forwarded only to B's link as before.
        
        \begin{tabular}{@{}lll@{}}
            \toprule
            MAC Address & Interface & Time \\ \midrule
            B & B & \textit{(i)} \\
            E & E & \textit{(ii)} \\
            A & A & \textit{(iii)} \\ \bottomrule
        \end{tabular}

        \item The frame is forwarded only to A's link because the switch took note of A's MAC address and its link when it sent a frame to B.
        
        \begin{tabular}{@{}lll@{}}
            \toprule
            MAC Address & Interface & Time \\ \midrule
            B & B & \textit{(iv)} \\
            E & E & \textit{(ii)} \\
            A & A & \textit{(iii)} \\ \bottomrule
        \end{tabular}
    \end{enumerate}
\end{enumerate}

\section*{Chapter 7}

\begin{enumerate}
    \item[P5.] \begin{enumerate}
        \item The protocol will not break down. When two stations, each associated with a different ISP, attempt to transmit at the same time, their will coordinate transmission with CSMA/CA. An ISP's AP may receive frames addressed the other AP because the wireless channel is a broadcast channel. These frames will be dropped by the AP. The channel's capacity will also be shared by two APs.
        \item There will be less collision and and each AP will be able to use the full capacity of its channel.
    \end{enumerate}
\end{enumerate}

\end{document}
