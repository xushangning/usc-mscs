\documentclass{article}

\usepackage{enumitem}
\setlist[enumerate, 2]{label=\alph*.}

\usepackage{siunitx}
\usepackage{booktabs}

\title{EE 450 Homework 4}
\author{Shangning Xu}

\begin{document}

\maketitle

\section*{Chapter 4}

\begin{enumerate}
    \item[P5.] \begin{enumerate}
        \item \begin{tabular}{@{}ll@{}}
            \toprule
            Prefix            & Link Interface \\ \midrule
            11100000 00       & 0              \\
            11100000 01000000 & 1              \\
            11100000          & 2              \\
            11100001 0        & 2              \\
            Otherwise         & 3              \\ \bottomrule
        \end{tabular}
        \item No prefix matches the first and the third address, so the datagrams are forwarded to link interface 3.
        
        Only 11100001 0 matches the second address, so the datagram is forwarded to link interface 2.
    \end{enumerate}

    \item[P6.] The destination IP range associated with interface 0 is 0000 0000 through 0011 1111, a total of 64 addresses.

    The range with interface 1 is 0100 0000 through 0101 1111, a total of 32 addresses.

    The range with interface 2 is 0110 0000 through 1011 1111, a total of 96 addresses.

    The range with interface 3 is 1100 0000 through 1111 1111, a total of 64 addresses.
    
    \item[P7.] The destination IP range associated with interface 0 is 1100 0000 through 1101 1111, a total of 32 addresses.
    
    The range with interface 1 is 1000 0000 through 1011 1111, a total of 64 addresses.

    The range with interface 2 is 1110 0000 through 1111 1111, a total of 32 addresses.

    The range with interface 3 is 0000 0000 through 0111 1111, a total of 128 addresses.

    \item[P8.] \begin{itemize}
        \item Subnet 1: 223.1.17.128/26
        \item Subnet 2: 223.1.17.0/25
        \item Subnet 3: 223.1.17.192/28
    \end{itemize}

    \item[P11.] An example of an IP address that can be assigned to the network is 128.119.40.129.
    
    The prefixes of the four subnets are 128.119.40.64/28, 128.119.40.80/28, 128.119.40.96/28 and 128.119.40.112/28.
 
    \item[P16.] \begin{enumerate}
        \item \begin{itemize}
            \item LAN-side interface of the router: 192.168.1.4
            \item Computers: 192.168.1.1, 192.168.1.2, 192.168.1.3
        \end{itemize}
        \item \begin{tabular}{@{}ll@{}}
            \toprule
            WAN Side & LAN Side \\ \midrule
            24.34.112.235:5001 & 192.168.1.1:3345 \\
            24.34.112.235:5002 & 192.168.1.1:3346 \\
            24.34.112.235:5003 & 192.168.1.2:3345 \\
            24.34.112.235:5004 & 192.168.1.2:3346 \\
            24.34.112.235:5005 & 192.168.1.3:3345 \\
            24.34.112.235:5006 & 192.168.1.3:3346 \\ \bottomrule
        \end{tabular}            
    \end{enumerate}
\end{enumerate}

\section*{Chapter 5}

\begin{enumerate}
    \item[P3.] \begin{tabular}{@{}llllllll@{}}
        \toprule
        step & $N'$ & $D(t),p(t)$ & $D(u),p(u)$ & $D(v),p(v)$ & $D(w),p(w)$ & $D(y),p(y)$ & $D(z),p(z)$ \\ \midrule
        0 & $x$ & $\infty$ & $\infty$ & $3,x$ & $6,x$ & $6,x$ & $8,x$ \\
        1 & $xv$ & $7,v$ & $6,v$ &  & $6,x$ & $6,x$ & $8,x$ \\
        2 & $xvw$ & $7,v$ & $6,v$ &  &  & $6,x$ & $8,x$ \\
        3 & $xvwy$ & $7,v$ & $6,v$ &  &  &  & $8,x$ \\
        4 & $xvwyu$ & $7,v$ &  &  &  &  & $8,x$ \\
        5 & $xvwyut$ &  &  &  &  &  & $8,x$ \\
        6 & $xvwyutz$ &  &  &  &  &  &  \\ \bottomrule
    \end{tabular}

    \item[P5.] \begin{tabular}{@{}lllll@{}}
        \toprule
        u & v & x & y & z \\ \midrule
        6 & 5 & 2 & 5 & 0 \\ \bottomrule
    \end{tabular}
\end{enumerate}

\end{document}
