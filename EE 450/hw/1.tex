\documentclass{article}

\usepackage{enumitem}
\setlist[enumerate, 2]{label=\alph*.}

\usepackage{siunitx}

\title{EE 450 Homework 1}
\author{Shangning Xu}

\begin{document}

\maketitle

\begin{enumerate}
    \item[P4.] \begin{enumerate}
        \item We can have 4 connections between each pair of adjacent switches, giving a total of 16 connections.
        \item 8, with 4 connections passing through the lower-left switch and 4 through upper-right.
        \item Yes. We can route 2 connections between A and C through B and the other 2 through D, and route connections between B and D through A and C.
    \end{enumerate}

    \item[P5.] \begin{enumerate}
        \item The textbook has shown that the transmission delay is \SI{2}{min} through each tollbooth, and the caravan propagates for a total of \SI{150}{km}, so the end-to-end delay is
        \[
            \SI{150}{km} / (\SI{100}{km/h}) + \SI{2}{min} \cdot 3 = \SI{96}{min}.
        \]
        \item With only eight cars, the transmission delay is now $8 / (\SI{5}{/min}) = \SI{1.6}{min}$, so the end-to-end delay is
        \[
            \SI{150}{km} / (\SI{100}{km/h}) + \SI{1.6}{min} \cdot 3 = \SI{94.8}{min}.
        \]
    \end{enumerate}

    \item[P6.] \begin{enumerate}
        \item $d_\textrm{prop} = m/s$ seconds
        \item $d_\textrm{trans} = L/R$ seconds
        \item $(L/R + m/s)$ seconds
        \item It is at A's end of the link.
        \item It is still in the link.
        \item The first bit has reached Host B.
        \item We have
        \[
            \frac{m}{2.5 \cdot 10^8} = \frac{120\textrm{ bits}}{56\textrm{ kbps}}.
        \]
        Solving for $m$, we get $m \approx 5.4 \cdot 10^5$.
    \end{enumerate}

    \item[P7.] The end-to-end delay for a packet is $\SI{56}{B} / \SI{2}{Mbps} + \SI{10}{ms} = \SI{10.224}{ms}$. However, bits in a packet are not created at the same time. The first bit in the packet is created $\SI{56}{B} / \SI{64}{kbps} = \SI{7}{ms}$ before the last bit. Therefore, 10.224--17.224 ms elapses from the time a bit is created until it is decoded.

    \item[P10.] The end-to-end delay is
    \[
        2d_\textrm{proc} + \sum_{i = 1}^3 \left(\frac{L}{R_i} + \frac{d_i}{s_i}\right).
    \]
    Substituting actual values into the formula, we obtain 64 ms for the end-to-end delay.

    \item[P12.] The queuing delay is $\SI{1500}{B} \cdot 4.5 / \SI{2}{Mbps} = \SI{27}{ms}$. Similarly, the queuing delay for the general case is $(nL + L - x)/R$.
    
    \item[P20.] $\min\{R_s, R_c, R/M\}$
    
    \item[P21.] If the server can only use one path, the maximum throughput is
    \[
        \max_{k = 1, \dots, M} \min_{i = 1, \dots, N} R_i^k.
    \]
    If the server can use all $M$ paths, the maximum throughput is
    \[
        \sum_{k = 1}^M \min_{i = 1, \dots, N} R_i^k.
    \]

    \item[P25.] \begin{enumerate}
        \item $R \cdot d_\textrm{prop} = \SI{2}{Mbps} \cdot \SI{20000}{km} / (\SI{2.5d8}{m/s}) = \SI{160}{kb}.$
        \item The number of bits that will be in the link is at its maximum when both B is receiving and A is transmitting. Because the first bit and the last bit in the link is $d_\textrm{prop}$ time apart when transmitted into the link, there are at most $R \cdot d_\textrm{prop} = \SI{160}{kb}$ in the link.
        \item The maximum possible number of bits in the link at any given time.
        \item The width of a bit is $\SI{20000}{km} / \num{160000} = \SI{125}{m}$, longer than an American football field at 360 feet (110 m).
        \item $m / (Rm/s) = s/R$
    \end{enumerate}

    \item[P28.] \begin{enumerate}
        \item The total transmission time is
        \[
            \SI{800000}{b} / \SI{2}{Mbps} + \SI{20000}{km} / (\SI{2.5d8}{m/s}) = \SI{0.48}{s}.
        \]
        \item The new total transmission time is now
        \[
            20(\SI{40000}{b} / \SI{2}{Mbps} + \SI{20000}{km} / (\SI{2.5d8}{m/s})) = \SI{2}{s}.
        \]
        \item After the file is broken up into packets, transmission time is now more than 4 times as long as before, because propagation of bits across different packets no longer overlaps, contributing to large increase in total propagation time.
    \end{enumerate}
\end{enumerate}

\end{document}
