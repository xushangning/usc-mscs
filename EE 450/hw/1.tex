\documentclass{article}

\usepackage{enumitem}
\setlist[enumerate, 2]{label=\alph*.}

\usepackage{siunitx}

\title{EE 450 Homework 1}
\author{Shangning Xu}

\begin{document}

\maketitle

\begin{enumerate}
    \item[P4.] \begin{enumerate}
        \item Because at most four hosts can be attached to each switch at any time, the maximum number of simultaneous connections is eight.
        \item 4
        \item No, we can't, because all four connections between A and C will use up all four circuits in each link connected to B or D, making it impossible to reserve circuits for connections between B and D.
    \end{enumerate}

    \item[P5.] \begin{enumerate}
        \item The total delay consists of the time to travel 150 km, two total nodal delays, and the time to clear the third toll booth. The textbook has shown that the total nodal delay is 62 min, so the total delay is
        \[
            \SI{150}{km} / (\SI{100}{km/h}) + \SI{62}{min} \cdot 2 + \SI{2}{min} = \SI{216}{min}.
        \]
        \item With only eight cars, the transmission delay is now $8 / (\SI{5}{/min}) = \SI{1.6}{min}$ and the total delay $\SI{60}{min} + \SI{1.6}{min} = \SI{61.6}{min}$. The total delay is
        \[
            \SI{150}{km} / (\SI{100}{km/h}) + \SI{61.6}{min} \cdot 2 + \SI{1.6}{min} = \SI{214.8}{min}.
        \]
    \end{enumerate}

    \item[P6.] \begin{enumerate}
        \item $d_\textrm{prop} = m/s$
        \item $d_\textrm{trans} = L/R$
        \item $L/R + m/s$
        \item It is at A's end of the link.
        \item It is still in the link.
        \item It is at B's end of the link.
        \item We have
        \[
            \frac{m}{2.5 \cdot 10^8} = \frac{120\textrm{ bits}}{56\textrm{ kbps}}.
        \]
        Solving for $m$, we get $m \approx 5.4 \cdot 10^5$.
    \end{enumerate}

    \item[P7.] The end-to-end delay for a packet is $\SI{56}{b} / \SI{2}{Mbps} + \SI{10}{ms} = \SI{10.028}{ms}$. However, bits in a packet are not created at the same time. The first bit in the packet is created $\SI{56}{b} / \SI{64}{kbps} = \SI{7.2}{ms}$ before the last bit. Therefore, 10.028--17.228 ms elapses from the time a bit is created until it is decoded.

    \item[P10.] The end-to-end delay is
    \[
        2d_\textrm{proc} + \sum_{i = 1}^3 \left(\frac{L}{R_i} + \frac{d_i}{s_i}\right).
    \]
    Substituting actual values into the formula, we obtain 48.25 ms for the end-to-end delay.

    \item[P12.] The queuing delay is $\SI{1500}{b} \cdot 4.5 / \SI{2}{Mbps} = \SI{3.375}{ms}$. Similarly, the queuing delay for the general case is $(nL + x)/R$.
    
    \item[P20.] $\min\{R_s, R_c, R/M\}$
    
    \item[P21.] If the server can only use one path, the maximum throughput is
    \[
        \min_{k = 1, \dots, M} \min_{i = 1, \dots, N} R_i^k.
    \]
    If the server can use all $M$ paths, the maximum throughput is
    \[
        \sum_{k = 1}^M \min_{i = 1, \dots, N} R_i^k.
    \]

    \item[P25.] \begin{enumerate}
        \item $R \cdot d_\textrm{prop} = \SI{2}{Mbps} \cdot \SI{20000}{km} / (\SI{2.5d8}{m/s}) = \SI{160}{kb}.$
        \item The number of bits that will be in the link is at its maximum when both B is receiving and A is transmitting. Because the first bit and the last bit in the link is $d_\textrm{prop}$ time apart when transmitted into the link, there are at most $R \cdot d_\textrm{prop} = \SI{160}{kb}$ in the link.
        \item The maximum possible number of bits in the link at any given time.
        \item The width of a bit is $\SI{20000}{km} / \num{160000} = \SI{125}{m}$, longer than an American football field at 360 feet (110 m).
        \item $m / (Rm/s) = s/R$
    \end{enumerate}

    \item[P28.] \begin{enumerate}
        \item The total transmission time is
        \[
            \SI{800000}{b} / \SI{2}{Mbps} + \SI{20000}{km} / (\SI{2.5d8}{m/s}) = \SI{0.48}{s}.
        \]
        \item The new total transmission time is now
        \[
            20(\SI{40000}{b} / \SI{2}{Mbps} + \SI{20000}{km} / (\SI{2.5d8}{m/s})) = \SI{2}{s}.
        \]
        \item After the file is broken up into packets, transmission time is now more than 4 times as long as before, because propagation of bits across different packets no longer overlaps, contributing to large increase in total propagation time.
    \end{enumerate}
\end{enumerate}

\end{document}
